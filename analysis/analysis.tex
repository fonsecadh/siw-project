\documentclass[11pt]{article}
\usepackage[T1]{fontenc}
\usepackage{lmodern}
\usepackage{parskip}
\usepackage[colorlinks=true,urlcolor=Blue,linkcolor=black,citecolor=black]{hyperref}
\usepackage{graphicx}
\usepackage{amsmath}
\usepackage[utf8]{inputenc}
\usepackage[spanish]{babel}
\usepackage{fancyhdr}
\usepackage{csquotes}
\usepackage{lastpage}
\usepackage{array}
\usepackage{listings}
\usepackage{color}
\definecolor{dkgreen}{rgb}{0,0.6,0}
\definecolor{gray}{rgb}{0.5,0.5,0.5}
\definecolor{mauve}{rgb}{0.58,0,0.82}
\usepackage[affil-it]{authblk}
\usepackage[activate={true,nocompatibility},final,tracking=true,kerning=true,spacing=true,factor=1100,stretch=10,shrink=10]{microtype}
\usepackage[hmargin=2cm,top=4cm,headheight=65pt,footskip=65pt]{geometry}

% Documento
\begin{document}
% Título
\title{We Are Data: Algorithms and the Making of Our Digital Selves. Un análisis.}
\author{Hugo Fonseca Díaz\\ \email{uo258318@uniovi.es}}
\affil{Escuela de Ingeniería Informática. Universidad de Oviedo.}
\maketitle
% Introducción
\section{Introducción}
En este documento realizaremos un análisis del libro \textit{We Are Data: Algorithms and the Making of Our Digital Selves}, de John Cheney-Lippold, profesor asistente de cultura americana en la Universidad de Michigan. Aunque ya había escrito varios artículos sobre la relación entre algoritmos y nuestras identidades o incluso nuestras vidas privadas, este es su primer libro sobre el tema (y su primer libro en general). Un libro que según su autor pretende conectar el constructo social que es el conocimiento con los cientos de capas técnicas que forman construcciones \textit{entrecomilladas} (hablaremos más tarde de esto) del mismo, realizadas por las grandes compañías mediante sus algoritmos.

Este análisis tiene cuatro principales secciones, consistentes en una tríada del tiempo (pasado, presente y futuro) y unas conclusiones finales. En las tres primeras secciones hablaremos de la base de pensamiento filosófico utilizada por el autor, de su análisis de la situación actual y de su visión del futuro. En la última sección realizaremos una serie de conclusiones sobre lo expuesto en el análisis.
% De dónde venimos
\section{De dónde venimos}
En esta sección expondremos el pensamiento filosófico de dos autores del siglo XX muy influyentes en nuestros días, Michel Foucault y Gilles Deleuze. Además, introduciremos el concepto de privacidad utilizado en los inicios de la constitución americana.

Antes de comentar a éstos dos autores, es importante que anotemos que pese a su gran influencia en el texto, no son las únicas inspiraciones de Cheney-Lippold, ya que sus conclusiones se apoyan en el trabajo de decenas de profesionales de diversos campos tanto de la informática como de ciencias sociales. Elegimos a éstos dos autores precisamente para destacar que aunque se habla mucho de éste tema a día de hoy, es algo que ya podemos encontrar (por supuesto no en los mismos términos) en Foucault y Deleuze. 
% Foucault
\subsection{Foucault, sociedades y biopolítica}
Foucault es uno de los filósofos más importantes hoy en día \cite{ernesto-foucault}. Nos centraremos en sus estudios sobre el sistema de prisiones y en su concepción de la biopolítica.

En su primer seminario titulado \textit{Lecciones sobre la voluntad de saber}, en las que Foucault va a tratar la cuestión de cómo en el discurso científico se articula al mismo tiempo pretensiones de justificación y pretensiones de dominio. Es decir, el conocimiento es poder en un sentido muy literal del término. Antes de la normalidad, los juicios eran puestas de verdad ante Dios, es decir, una concepción de la justicia como duelo entre individuos. Es en la modernidad donde el poder judicial es confiscado por el Estado. La monarquía ya no se apoya en el ejercicio de la justicia como justicia entre partes, sino que concibe la justicia como un reino universal impuesto desde arriba por los gobernantes. Dicha imposición toma la forma del viejo modelo eclesíastico de control individual, la Inquisición, un proceso de búsqueda de la verdad por un conjunto de documentación muy abundante donde se busca que el sujeto enuncie su propia verdad. El modelo del inquirir es la esencia del modelo de justicia impuesto por las instituciones modernas \cite{ernesto-foucault}. 

El Estado moderno no solamente busca una nueva forma de justicia, que es la inquisición o la investigación, sino que también busca una nueva forma de conocimiento. Frente al conocimiento sistemático o escolástico propio de las universidades, la época moderna como una época articulada bajo la razón de Estado busca un \textbf{conocimiento clasificador}. Lo que el Estado quiere saber no es el sistema de todo el mundo, sino que quiere tener clasificados y burocráticamente determinados a los individuos en sus clases (clases sociales, clases de género, pero también clases botánicas, clases zoológicas, etc). Así, según Foucault, el paradigma del conocimiento estatal es la \textbf{enciclopedia}. En la enciclopedia el conocimiento no está estructurado sistemáticamente sino que está perfectamente ordenado alfabéticamente. El empirismo sería una doctrina típicamente estatal que busca conocer la realidad tal y como es y no como debería ser según nuestras propias teorías o según la razón en sentido abstracto \cite{ernesto-foucault}.

Una vez estudiada como funciona la ciencia del Estado, la estadística (también surgen la sociología, la psiquiatría o la economía), Foucault analiza cuáles son los mecanismos que tiene esa sociedad de regularse punitivamente. Según Foucault hay cuatro procedimientos de actividad punitiva, la exclusión, la confiscación, la señalización y el encierro. En la época clásica no se castiga actos sino a personas. Los únicos que están excluidos de este procedimiento son los locos, puesto que los locos no tienen \textit{yo}. La gran pregunta es como es posible que se produjera una evolución de las conductas penales para que de cuatro conductas posibles actualmente se entienda que la forma máxima de la pena es el encierro. La respuesta que nos da Foucault es la siguiente, en la época moderna el delito no se contempla desde un punto de vista moral, es simplemente una violación de la ley civil. La función de la pena ya no es la reinserción o la reinstitución de un orden sagrado, sino simplemente una cuestión preventiva. Sin embargo esto nos plantea una duda, ¿por qué la prisión? Si la función es prevenir, por qué ya no hay ejecuciones expectaculares en público como se hacía antaño. Para responder a ésto Foucault hace un gran recorrido histórico para llegar a la conclusión de que el conjunto de la sociedad es una prisión. Es decir, la prisión no es sino una institución más junto a la Iglesia, la Escuela, la Familia, de normalización y de serialización de la conducta individual \cite{ernesto-foucault}.

Con la revolución francesa llegó una nueva forma de entender la política como una concepción anatómica, donde los individuos forman parte orgánica del Estado, es decir, son miembros (en el sentido más literal del término) del Estado, de la misma forma que un brazo es un miembro del cuerpo. El ojetivo de esta nueva política anatómica es disciplinar a los individuos de la misma forma que un individuo disciplina su propio término. Juega aquí una parte importante la propia palabra \textit{disciplina}, que al mismo tiempo refiere a la actividad física y a la científica. Son estas disciplinas, las ciencias humanísticas, las que articulan una serie de formas de distribución de individuos en el espacio, control de la actividad, organización de la población, etc. Todo con la finalidad de crear una norma estadística \cite{ernesto-foucault}.

Foucault hace aquí una distinción entre la época antigua, en la que la sociedad estaba principalmente regulada por la ley, y la época moderna en la que la sociedad estaba principalmente regulada por la norma. La ley se aplica a las personas pero siempre desde un punto de vista exterior y esencialmente con motivo de una ofensa, mientras que la norma es mucho mas perversa e invasiva, pretendiendo constituir tu propia subjetividad buscando la interiorización de los comportamientos individuales en busca de una conducta específica. La norma a diferencia de la ley busca la constitución de sujetos normales y por exclusión de los anormales \cite{ernesto-foucault}.

Como explicamos antes, para Foucault la cárcel no es más una institución más junto a la Iglesia, la Escuela, la Fábrica o el Hospital que pretende normalizar los procedimientos de reproducción, educación, producción y supervivencia de los individuos. Esto se consigue a través de tres mecanismos de control y de disciplina a saber la observación jerárquica, el juicio normalizador y el examen (siendo éste último la síntesis de los dos mecanismos anteriores). El examen no solo evalua tu conocimiento de la realidad sino que te pretende definir como un sujeto más o menos óptimo respecto de un patrón ideal, véase el alumno ideal, el enfermo ideal o el encarcelado ideal. El poder moderno tiene un control disciplinario, es decir, no simplemente se ocupa de aquello que las personas hacen sino de aquello que dejan de hacer, y por tanto se articula bajo la dinámica entre la insatisfacción constante y el estándar absoluto. El objetivo de la disciplina estatal moderna es corregir el comportamiento desviado, es decir, cumplir que el sujeto que se ha desviado (debido a la enfermedad, a los malos estudios, etc) regrese a ese estándar de conducta. La disciplina se impone a través de normas muy precisas y detalladas de normalización, siendo en ese momento en el que se constituye la diferencia entre lo normal y lo anormal. Los objetivos de la verdad y del poder son los mismos, \textit{conocer es poder} en un sentido muy absoluto. El examen ubica a los individuos en lo que Foucault denomina un \textit{campo de documentación}, que son toda esa información que el Estado acumula de nosotros. Un ejemplo de las consecuencias fatales de dicha información categorizadora la encontramos en la Segunda Guerra Mundial. En los Estados donde había una categorización previa de la población judía fue donde se llevó a cabo un exterminio prácticamente total de dicha población por parte de los nazis. Lo mismo ocurre hoy en día con Internet, la información extraída de nuestra actividad en la red resulta en un conocimiento que refuerza esas formas de control y de poder, que a su vez resulta en un disciplinamiento a nosotros mismos interiorizando la constante vigilancia de nuestros pares. \cite{ernesto-foucault}.

Para Foucault el tema central es el de la visibilidad, siendo el \textit{panóptico} el emblema de la misma en la transición de la antigüedad a la modernidad. El panóptico es un tipo de cárcel en la que desde una torre central se podría vigilar a todas las celdas, que tendrían una pared trasparente orientada hacia esa torre. Lo importante no es que los presos sean constantemente vigilados por ese individuo, sino que los presos ante la expectativa de la posible vigilancia ya se regulan y se vigilan a sí mismos. En las redes sociales, es el miedo al linchamiento digital lo que hace que los individuos se automoderen y se censuren sin necesidad de que exista realmente una censura estatal. El objetivo sería constituir cuerpos dóciles, prisioneros, soldados, trabajadores, escolares y a día de hoy \textit{influencers}, que sean útiles y a la vez fáciles de controlar \cite{ernesto-foucault}.

Foucault también estudia y analiza el poder soberano, un tipo de poder ejercido desde el Estado y las instituciones de gobierno. Esta forma de poder se articula bajo una forma específica histórica que es la del derecho a matar. El poder soberano en la época clásica es un poder deductivo, un poder que substrae cosas de los individuos que son sus súbditos. El Estado tiene un poder soberano porque puede apropiarse o deducirte la riqueza, la fuerza de trabajo o en última instancia tu vida a través de las levas militares. En definitiva, es un poder que hace morir y deja vivir. Pero con esta forma de poder soberano y deductivo dice Foucault que a partir del siglo XVIII surge un nuevo poder que ya no busca hacer morir y dejar vivir, sino todo lo contrario, \textbf{dejar morir y hacer vivir}. Ésto es lo que él denomina el biopoder o \textbf{la biopolítica}. Estas políticas son aquellas que lleva el Estado como la promoción de la reproducción, mantenimiento de unos estándares de sanidad, institución de plataformas educativas, etc. Es el momento en el que el Estado no simplemente reprime o controla a la población sino que la quiere moldear a su imagen y semejanza para que sean sujetos perfectamente controlables, predecibles, sumisos, etc.
% Deleuze
\subsection{Deleuze y las sociedades de control}
En esta sección hablaremos del ensayo de Gilles Deleuze titulado \textit{Postscript on the societies of control}. En esta corta obra Deleuze argumenta la existencia de un nuevo tipo de sociedad, la sociedad de control. Dicha sociedad se contrasta con las sociedades de disciplina expuestas por Foucault y de las que hablamos en la sección anterior. Sin embargo, es importante destacar que Deleuze no piensa que estemos abandonando por completo dichas sociedades de disciplina, simplemente estamos en período transitorio \cite{cp-societies}.

En la sociedad de control, el ejercicio del poder ya no requiere encerrar a la población en espacios como la Escuela, la Fábrica o la Prisión. Los regímenes disciplinarios abandonan dichos espacios fijos y se expanden al conjunto de la sociedad. Por ejemplo, los datos recolectados mediante nuestros smartphones son utilizados por corporaciones y gobiernos. Si en una sociedad disciplinaria el estado normalizaba tu comportamiento, en una sociedad de control eres libre de hacer lo que quieras por ejemplo con los mencionados teléfonos, ya que cuanta más actividad realices más información tendrá el Estado para vigilarte, evaluarte y realizar operaciones con tus datos. Otro ejemplo, el de los aeropuertos con reconocimiento facial, implica que cuanto más viajes, más datos tendrá el reconocimiento facial para identificarte correctamente, así como información sobre tus hábitos de transporte. Mientras en una sociedad disciplinaria el Estado te fuerza a quedarte en espacios definidos y a restringir tu movimiento para poder vigilarte, en la sociedad de control el Estado de invita a moverte libremente para recoger más información sobre ti \cite{cp-societies}.

El problema viene cuando dichas formas de control ya no son percibidas como tal, sino como un ejercicio de tu libertad. Es ésto lo que hace que las sociedades de control sean tan peligrosas y a su vez efectivas. Cuando compro un nuevo teléfono, hago una cuenta en Facebook, realizo un test de ADN, compro una tarjeta de crédito o voy a un aeropuerto, lo estoy haciendo todo por mi propia voluntad, pero gracias a estas acciones estoy aumentando el poder que las instituciones tienen sobre mi. En cuanto cometa algun crimen o realice una acción no apropiada para dichas instituciones, será muy fácil limitar mi libertad, ponerme restricciones o castigarme desde la distancia. Como tenemos una gran dependencia de las tarjetas electrónicas, cuentas en Internet, código o servicios móviles que pueden ser desactivados en cualquier momento, no es necesario una intervención física para que el poder sea ejercido \cite{cp-societies}.

Mientras que en las sociedades disciplinarias los individuos se movían de la Escuela a la Fábrica, de la Fábrica a la Prisión o al Hospital, etc, en las sociedades de control no necesitas salir de una para entrar en otra. Como dice Deleuze, \textit{así como la corporación sustitituye a la Fábrica, el entrenamiento perpetuo sustituye a la Escuela, y el control continuo sustituye al Examen}. Como en las sociedades disciplinarias dichas instituciones estaban confinadas a un espacio concreto, al menos sabías cuando estabas saliendo de una, por ejemplo al salir del trabajo. Esto ya no es así en la sociedad de control, incluso si estás en casa se espera de ti un continuo estado de trabajo, comprobando emails, estando atento a llamadas de la empresa, etc \cite{cp-societies}.

Para Deleuze \textbf{ya no somos individuos, somos dividuos}. El origen de la palabra \textit{individuo} viene de lo que no es divisible, algo que no puede dividirse. En las sociedades de disciplina existía el concepto del individuo, normalmente asignándoles un número en las distintas instituciones. En las sociedades de control no es cierto el que no podamos ser divididos, de hecho estamos constantemente siendo divididos en demográficas, muestras de marketing, resultados de encuestas, estadísticas. En resumen, somos un agregado infinito de divisiones por las cuales las instituciones nos observan. El número se replaza por el código.
% Privacidad pasada
\subsection{Privacidad pasada}
¿Qué fue la privacidad? En Estados Unidos, la privacidad se consideró como el derecho del individuo a determinar con qué extensión debían sus sentimientos, pensamientos y emociones ser comunicados a otros. Nunca debía ser obligado a expresarlos (excepto al ser testigo en un juicio) y en caso de que escogiera expresarlos, debe tener el poder de fijar los límites de la publicidad que se les da. También se expresó como "el derecho a estar solo".

Recalca el autor que esta privacidad pasada que desde nuestra situación actual parece tan idílica estaba lejos de serlo, ya que en Estados Unidos la privacidad ya estaba muerta desde el comienzo para las mujeres y las personas no blancas. El autor cita a la feminista Catharine MacKinnon que modifica la frase expresada en el párrafo anterior a "el derecho del hombre a estar solo", puesto que a veces se ejercía el derecho de la privacidad del hombre a costa de las mujeres. Tampoco existe la privacidad para los menores hasta que cumplen la mayoría de edad, y ciertos grupos como las personas mayores o las personas con discapacidad tienen una privacidad siempre condicionada. Por último, la privacidad tampoco ha existido nunca para la gente pobre, las que no disponen de un hogar o las que no tienen dinero para pagar a un abogado en caso de invasión de la privacidad.
% Dónde estamos
\section{Dónde estamos}
En esta sección repasaremos el análisis que realiza Cheney-Lippold de la situación que vivimos actualmente, hablaremos de los tipos medibles y de como la biopolítica afecta a nuestros dividuos. Por último repasaremos lo que el autor considera como la privacidad actual y cómo en ciertos aspectos está obsoleta.
% Tipos medibles
\subsection{Tipos medibles}
Los tipos medibles son en última instancia clasificaciones, observadas empíricamente y convertidas a datos. Los modelos de tipo medible son objetos que representan datos y que a su vez determinan los parámetros discursivos de quien (y quien no) podemos ser. El resultado es una serie categorías de tipos medibles a las que nuestros datos pertenecen o no. Eso es lo que indica como los sistemas de clasificación nos categorizan algorítmicamente cuando estamos en red.

Por ejemplo, aunque yo me considere un hombre, puede que debido a las acciones que realice en determinadas páginas o a los sitios web que visite, Google me categorice como una "mujer". Eso implica que a partir de ese momento todo lo que haga será interpretado como lo que haría una "mujer" de Google, si visito un blog de un amigo ese amigo podrá observar que sus últimas publicaciones están atrayendo la atención de más mujeres, así que quizás siga publicando en esa línea o por el contrario decida cambiarla. Todo basado en la premisa falsa de que quien está entrando a su blog no son mujeres sino "mujeres". Esta mala categorización el autor la define como \textit{neocategorización}, ya que mientras me comporte como una "mujer" a Google le da igual que realmente sea un hombre o una mujer.

Esto implica que los tipos medibles no están escritos en piedra, lo que sea una "mujer" hoy puede cambiar mañana, es algo inestable. Si los "hombres" pasan a visitar mucho el blog de mi amigo y vuelvo a entrar en otra ocasión puede que Google piense que soy menos "mujer" y más "hombre". El problema es que no sabemos realmente cómo es una "mujer" o un "hombre" de Google, ya que sus algoritmos de clasificación son privativos, y peor aún, no sabemos siquiera que "seres" somos cuando estamos en red. Puede que haya más categorías que ni siquiera sabemos que existan, por seguir con el ejemplo de antes, si mi amigo tiene un blog hablando de sus experiencias como persona homosexual, puede que al entrar en dicho blog se me adhiera una nueva clasificación de "homosexual", incluso si no lo soy realmente. También esto afectaría a los sitios que visite a continuación, ya que pensarán que su contenido atrae a más "homosexuales".

Quien somos online nos es desconocido, al final del día somos un agregado infinito de tipos medibles que adquirimos al realizar nuestras actividades rutinarias, los famosos dividuos de Deleuze de los que hablamos al inicio del análisis.
% Biopolítica de los dividuos
\subsection{Biopolítica de los dividuos}
La biopolítica de Foucault, recordemos que era el poder de dejar morir y hacer vivir, también afecta a nuestros dividuos. Cheney-Lippold define dos tipos distintos de biopolíticas, las biopolíticas duras y las biopolíticas blandas.

Las biopolíticas duras son las que definimos previamente, las que el Estado ejerce en temas sociales, reproducción, sanidad, género, clase, raza, edad, etc. Las biopolíticas blandas son las que construyen categorías algorítmicas como "reproducción", "sanidad", "género", "clase", "raza", etc. Es decir, las biopolíticas blandas son las que lidian con nuestros dividuos.

Ponemos ahora un ejemplo real de biopolítica blanda, la empleada por \textit{Google Flu Trends} en 2009. Realizando estudios sobre los millones de consultas recibidas a su buscador, Google Flu Trends observaba las que tenían que ver con enfermedades varias, tipos de medicina y tipos de tratamiento alternativos. Todo ésto era un buen indicador de la gripe para Google, así que crearon un tipo medible "gripe" que categorizaba si tenías gripe o no. Por supuesto, ésto era lo que querían, pero realmente Google Flu Trends solo podía saber si tenías la "gripe", no la gripe.  

Un conjunto de unas 160 consultas indicativas pasaron a ser la nueva definición de la gripe. Con ésto Google cambió por completo el significado real de la gripe. Al crear la "gripe", ya no importa si realmente tienes la nariz mocosa o si tienes fiebre, solo importa si buscas en google repetidamente "cómo saber si tengo fiebre" o "nariz mocosa".

Como Google asigna la "gripe" a un comportamiento en la web y no a una persona con síntomas reales, la respuesta estatal americana a una epidemia de gripe pasa también a depender de la definición de "gripe" de Google y de cómo sus algoritmos reconocían actividad de gripe en una determinada zona. Comercialmente, diversas compañías como Vicks contrataron personal de marketing que anunciara productos usando los datos obtenidos con la Google Flu Trends, es decir, la "gripe" ya no es una categorización experimental, es una categoría procesable que requiere de precisión, por lo que era poderosa y rentable.

Sin embargo, Google Flu Trends falló estrepitosamente, solo casi la mitad de las predicciones ocurrieron realmente. Esto no es ninguna sorpresa por lo que explicábamos antes, Google no estaba realmente midiendo la gripe, sino la "gripe". Puedes estar encerrado en un ascensor todo el día con una persona infectada de la "gripe" sin miedo a contagiarte, a menos que tenga la gripe claro. El problema es que no importa la eficacia real de la "gripe", la "gripe" era considerada importante. Se gastó dinero en anuncios, las noticias y oficiales del Estado hablaban de la "gripe" como si fuera real, es decir, la "gripe" existía en tu realidad lo quisieras o no.

Esto lleva a otros problemas, si tú y tus amigos buscábais en Google "vacaciones de primavera" ("spring break"), al estar relacionado con la "gripe" puede que las autoridades cerraran el colegio de vuestra ciudad. Otro ejemplo, si tenías la "gripe" (que no la gripe), podías no ver anuncios de conciertos o actividades multitudinarias en tu ciudad por miedo a que provocaras un contagio masivo.

Para finalizar esta subsección, pensar en que si algo como la "gripe" de Google puede ser problemática y afectar a tu vida privada real de forma directa, que pasaría si superpotencias mundiales crearan tipos medibles como "antigobierno", "revolucionario", etc. No debemos subestimar los daños que pueden causar este tipo de categorizaciones en la red.
% Privacidad presente
\subsection{Privacidad presente}
¿Qué es la privacidad? Para Cheney-Lippold, hoy en día la privacidad no significa desconectarse del mundo o esconder nuestras debilidades para que no las vea un ojo vigilante. Es nuestro derecho a "mantener la distancia sobre otros mediante la preservación de nuestra privacidad corporal y la integridad de nuestro ser", tal como lo expresa el teórico social Anthony Giddens, citado por el autor. 

Dentro del territorio de nuestro ser, sabemos qué se nos ordena, que consecuencias tienen nuestros actos y por tanto podemos determinar lo que significa \textit{ser} nosotros. Sin embargo, si desconocemos las cuestiones listadas anteriormente para con nuestros dividuos, carecemos del derecho de la privacidad tal como describimos al inicio de la sección.

Para los líderes tecnológicos, la privacidad importa bien poco, y no tienen ningún reparo en reconocerlo. Scott McNealy, CEO de Sun Microsystems, expresaba en 1999 que "Tienes cero privacidad de todas formas. Superalo." Para Mark Zuckerberg, creador de Facebook, la privacidad "ya no es una norma social", y para el antiguo CEO de Google Eric Schmidt la privacidad no solo es innecesaria sino que encima es peligrosa, ya que "si estás haciendo algo que no quieres que nadie más sepa, no deberías hacerlo en primer lugar." Por supuesto esta "guerra" contra la privacidad no es más que un enfrentamiento de intereses entre sus negocios y los derechos de la gente, ya que la privacidad les afecta directamente.

Tal como habíamos comentado en la sección \textit{Privacidad pasada}, la privacidad nunca ha existido para ciertos grupos de personas. Sin embargo, esta es la primera vez que los hombres blancos de clase trabajadora se encuentran dentro del control inherente a la vigilancia causada por la invasión de Internet en la vida privada. Ahora bien, según comenta el autor, estas cuestiones de vigilancia y privacidad afectan a nuestro mundo social sin importar el género, raza o clase. El poder nos afecta a todos y es la privacidad la forma que tenemos de regular el cómo lo hace, por lo que es un derecho que cuenta con un gran rango de partidarios.

Esa idea idílica de privacidad realmente no ha muerto del todo, podemos votar (o no), leer un libro en la biblioteca o hablar con quien nos de la gana. Sin embargo, cuando conocemos el contexto en el que estamos, la privacidad nos brinda mucho más que decisiones de acción o inacción, lo que le recuerda al autor al concepto de "privacidad como integridad contextual" del filósofo Helen Nissembaum. Con la privacidad, podemos conocer y entendernos a nosotros mismos, tener una "integridad del ser" como expresaba Anthony Giddens. Mientras que la privacidad liberal abogaba por una "esfera privada", la privacidad como integridad contextual aborda las distintas complejidades de la vida social en red en la cual el poder no nos abandona en ningún momento.

También es la privacidad la que nos permite ser hedonistas, vagos o ignorar las normas sociales en los contextos apropiados, el saber que podemos hacer las cosas de forma distinta en un espacio privado ya que no habrá nadie para juzgarnos. Es gracias a estas desviaciones de la norma por las cuales podemos mantener nuestra integridad del ser.

Para el autor, la privacidad liberal tal y como la conocíamos ya no puede garantizarnos esa protección, es necesaria una nueva forma de privacidad, la privacidad de los dividuos. Cuando estamos compuestos por datos, no tenemos una integridad del ser. Mientras que si andamos por la calle podemos disfrutar de lo que Nissenbaum llama "privacidad en público", es decir, sabemos cómo nuestro cuerpo, nuestro género y nuestra raza generan un sentido que necesitamos para poder saber como vivir, nuestros dividuos no generan un sentido que podamos comprender y por tanto perdemos nuestra integridad del ser.
% Hacia dónde deberíamos ir
\section{Hacia dónde deberíamos ir}
Aquí expondremos la idea de privacidad que el autor tiene de cara al futuro y de sus aplicaciones inmediatas y a largo plazo.
% Privacidad futura
\subsection{Privacidad futura}
¿Qué será la privacidad? La privacidad futura es la privacidad de los dividuos, el erudito en derecho de comunicación Jisuk Woo la define como "el derecho a no ser identificado". Cheney-Lippold recalca que en este caso no ser identificado significa no ser identificado personalmene ni algorítmicamente. Según el erudito en medios de comunicación y el filósofo Hellen Nissembaum este modo de privacidad de la no identificación se práctica de la mejor forma mediante la ofuscación. La ofuscación es la "adición deliberada de información ambigua, confusa o engañosa para interferir con la vigilancia y recopilación de datos".
% Ghost in the shell
\section{Ghost in the shell}
Ghost in the shell, Ghosts in the machine, Derrida, Hauntología, Binario Individuo/Dividuos, Ghosts in the dividuals.
% Bibliografía
\begin{thebibliography}{8}
    \bibitem{ernesto-foucault}
        Clase sobre Michel Foucault impartida por Ernesto Castro, profesor de filosofía de la Universidad Complutense de Madrid, \url{https://youtu.be/uVIt4MX7kvU}. Última vez accedido 18 de enero de 2021.

    \bibitem{cp-societies}
        Vídeo sobre las sociedades de control, realizado por el graduado en filosofía Jonas Ceika (seudónimo), \url{https://youtu.be/B_i8_WuyqAY}. Última vez accedido 18 de enero de 2021.
\end{thebibliography}
\end{document}


