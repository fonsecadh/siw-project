\documentclass[11pt]{article}
\usepackage[T1]{fontenc}
\usepackage{lmodern}
\usepackage{parskip}
\usepackage[colorlinks=true,urlcolor=Blue,linkcolor=black,citecolor=black]{hyperref}
\usepackage{graphicx}
\usepackage{amsmath}
\usepackage[utf8]{inputenc}
\usepackage[spanish]{babel}
\usepackage{fancyhdr}
\usepackage{csquotes}
\usepackage{lastpage}
\usepackage{array}
\usepackage{listings}
\usepackage{color}
\definecolor{dkgreen}{rgb}{0,0.6,0}
\definecolor{gray}{rgb}{0.5,0.5,0.5}
\definecolor{mauve}{rgb}{0.58,0,0.82}
\usepackage[affil-it]{authblk}
\usepackage[activate={true,nocompatibility},final,tracking=true,kerning=true,spacing=true,factor=1100,stretch=10,shrink=10]{microtype}
\usepackage[hmargin=2cm,top=4cm,headheight=65pt,footskip=65pt]{geometry}

% Documento
\begin{document}
% Título
\title{We Are Data: Algorithms and the Making of Our Digital Selves. Un análisis.}
\author{Hugo Fonseca Díaz\\ \email{uo258318@uniovi.es}}
\affil{Escuela de Ingeniería Informática. Universidad de Oviedo.}
\maketitle
% Introducción
\section{Introducción}
En este documento realizaremos un análisis del libro \textit{We Are Data: Algorithms and the Making of Our Digital Selves}, de John Cheney-Lippold, profesor asistente de cultura americana en la Universidad de Michigan. Aunque ya había escrito varios artículos sobre la relación entre algoritmos y nuestras identidades o incluso nuestras vidas privadas, este es su primer libro sobre el tema (y su primer libro en general). Un libro que según su autor pretende conectar el constructo social que es el conocimiento con los cientos de capas técnicas que forman construcciones \textit{entrecomilladas} (hablaremos más tarde de esto) del mismo, realizadas por las grandes compañías mediante sus algoritmos.

Este análisis tiene cuatro principales secciones, consistentes en una tríada del tiempo (pasado, presente y futuro) y unas conclusiones finales. En las tres primeras secciones hablaremos de la base de pensamiento filosófico utilizada por el autor, de su análisis de la situación actual y de su visión del futuro. En la última sección realizaremos una serie de conclusiones sobre lo expuesto en el análisis.
% De dónde venimos
\section{De dónde venimos}
En esta sección expondremos el pensamiento filosófico de dos autores del siglo XX muy influyentes en nuestros días, Michel Foucault y Gilles Deleuze. Además, expondremos el concepto de privacidad utilizado antes de vivir en la era de la información.

Antes de comentar a éstos dos autores, es importante que anotemos que pese a su gran influencia en el texto, no son las únicas inspiraciones de Cheney-Lippold, ya que sus conclusiones se apoyan en el trabajo de decenas de profesionales de diversos campos tanto de la informática como de ciencias sociales. Elegimos a éstos dos autores precisamente para destacar que aunque hay muchos profesionales y autores hablando de éste tema a día de hoy, es algo que ya podemos encontrar (por supuesto no en los mismos términos) en Foucault y Deleuze. 
% Foucault
\subsection{Foucault, sociedades y biopolítica}
% Deleuze
\subsection{Deleuze y las sociedades de control}
% Privacidad pasada
\subsection{Privacidad pasada}
¿Qué fue la privacidad? En Estados Unidos, la privacidad se consideró como el derecho del individuo a determinar con qué extensión debían sus sentimientos, pensamientos y emociones ser comunicados a otros. Nunca debía ser obligado a expresarlos (excepto al ser testigo en un juicio) y en caso de que escogiera expresarlos, debe tener el poder de fijar los límites de la publicidad que se les da. También se expresó como "el derecho a estar solo".

Recalca el autor que esta privacidad pasada que desde nuestra situación actual parece tan idílica estaba lejos de serlo, ya que en Estados Unidos la privacidad ya estaba muerta desde el comienzo para las mujeres y las personas no blancas. El autor cita a la feminista Catharine MacKinnon que modifica la frase expresada en el párrafo anterior a "el derecho del hombre a estar solo", puesto que a veces se ejercía el derecho de la privacidad del hombre a costa de las mujeres. Tampoco existe la privacidad para los menores hasta que cumplen la mayoría de edad, y ciertos grupos como las personas mayores o las personas con discapacidad tienen una privacidad siempre condicionada. Por último, la privacidad tampoco ha existido nunca para la gente pobre, las que no disponen de un hogar o las que no tienen dinero para pagar a un abogado en caso de invasión de la privacidad.
% Dónde estamos
\section{Dónde estamos}
% Tipos medibles
\subsection{Tipos medibles}
% Dividuos e individuos
\subsection{Dividuos e individuos}
% Biopolítica de los dividuos
\subsection{Biopolítica de los dividuos}
% Privacidad presente
\subsection{Privacidad presente}
¿Qué es la privacidad? Para Cheney-Lippold, hoy en día la privacidad no significa desconectarse del mundo o esconder nuestras debilidades para que no las vea un ojo vigilante. Es nuestro derecho a "mantener la distancia sobre otros mediante la preservación de nuestra privacidad corporal y la integridad de nuestro ser", tal como lo expresa el teórico social Anthony Giddens, citado por el autor.
% Hacia dónde deberíamos ir
\section{Hacia dónde deberíamos ir}
% Privacidad futura
\subsection{Privacidad futura}
¿Qué será la privacidad? La privacidad futura es la privacidad de los dividuos, el erudito en derecho de comunicación Jisuk Woo la define como "el derecho a no ser identificado". Cheney-Lippold recalca que en este caso no ser identificado significa no ser identificado personalmene ni algorítmicamente. Según el erudito en medios de comunicación y el filósofo Hellen Nissembaum este modo de privacidad de la no identificación se práctica de la mejor forma mediante la ofuscación. La ofuscación es la "adición deliberada de información ambigua, confusa o engañosa para interferir con la vigilancia y recopilación de datos".
% Ghost in the shell
\section{Ghost in the shell}
% Bibliografía
\begin{thebibliography}{8}
\end{thebibliography}
\end{document}


