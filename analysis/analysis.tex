\documentclass[11pt]{article}
\usepackage[T1]{fontenc}
\usepackage{lmodern}
\usepackage{parskip}
\usepackage[colorlinks=true,urlcolor=Blue,linkcolor=black,citecolor=black]{hyperref}
\usepackage{graphicx}
\usepackage{amsmath}
\usepackage[utf8]{inputenc}
\usepackage[spanish]{babel}
\usepackage{fancyhdr}
\usepackage{csquotes}
\usepackage{lastpage}
\usepackage{array}
\usepackage{listings}
\usepackage{color}
\definecolor{dkgreen}{rgb}{0,0.6,0}
\definecolor{gray}{rgb}{0.5,0.5,0.5}
\definecolor{mauve}{rgb}{0.58,0,0.82}
\usepackage[affil-it]{authblk}
\usepackage[activate={true,nocompatibility},final,tracking=true,kerning=true,spacing=true,factor=1100,stretch=10,shrink=10]{microtype}
\usepackage[hmargin=2cm,top=4cm,headheight=65pt,footskip=65pt]{geometry}

% Documento
\begin{document}
% Título
\title{We Are Data: Algorithms and the Making of Our Digital Selves. Un análisis.}
\author{Hugo Fonseca Díaz\\ \email{uo258318@uniovi.es}}
\affil{Escuela de Ingeniería Informática. Universidad de Oviedo.}
\maketitle
% Introducción
\section{Introducción}
En este documento realizaremos un análisis del libro \textit{We Are Data: Algorithms and the Making of Our Digital Selves}, de John Cheney-Lippold, profesor asistente de cultura americana en la Universidad de Michigan. Aunque ya había escrito varios artículos sobre la relación entre algoritmos y nuestras identidades o incluso nuestras vidas privadas, este es su primer libro sobre el tema (y su primer libro en general). Un libro que según su autor pretende conectar el constructo social que es el conocimiento con los cientos de capas técnicas que forman construcciones \textit{entrecomilladas} (hablaremos más tarde de esto) del mismo, realizadas por las grandes compañías mediante sus algoritmos.

Este análisis tiene cuatro principales secciones, consistentes en una tríada del tiempo (pasado, presente y futuro) y unas conclusiones finales. En las tres primeras secciones hablaremos de la base de pensamiento filosófico utilizada por el autor, de su análisis de la situación actual y de su visión del futuro. En la última sección realizaremos una serie de conclusiones sobre lo expuesto en el análisis.
% De dónde venimos
\section{De dónde venimos}
En esta sección expondremos el pensamiento filosófico de dos autores del siglo XX muy influyentes en nuestros días, Michel Foucault y Gilles Deleuze. Además, introduciremos el concepto de privacidad utilizado antes de vivir en la era de la información.

Antes de comentar a éstos dos autores, es importante que anotemos que pese a su gran influencia en el texto, no son las únicas inspiraciones de Cheney-Lippold, ya que sus conclusiones se apoyan en el trabajo de decenas de profesionales de diversos campos tanto de la informática como de ciencias sociales. Elegimos a éstos dos autores precisamente para destacar que aunque se habla mucho de éste tema a día de hoy, es algo que ya podemos encontrar (por supuesto no en los mismos términos) en Foucault y Deleuze. 
% Foucault
\subsection{Foucault, sociedades y biopolítica}
Foucault es uno de los filósofos más importantes hoy en día \cite{ernesto-foucault}. Nos centraremos en sus estudios sobre el sistema de prisiones y en su concepción de la biopolítica.

En su primer seminario titulado \textit{Lecciones sobre la voluntad de saber}, en las que Foucault va a tratar la cuestión de cómo en el discurso científico se articula al mismo tiempo pretensiones de justificación y pretensiones de dominio. Es decir, el conocimiento es poder en un sentido muy literal del término. Antes de la normalidad, los juicios eran puestas de verdad ante Dios, es decir, una concepción de la justicia como duelo entre individuos. Es en la modernidad donde el poder judicial es confiscado por el Estado. La monarquía ya no se apoya en el ejercicio de la justicia como justicia entre partes, sino que concibe la justicia como un reino universal impuesto desde arriba por los gobernantes. Dicha imposición toma la forma del viejo modelo eclesíastico de control individual, la Inquisición, un proceso de búsqueda de la verdad por un conjunto de documentación muy abundante donde se busca que el sujeto enuncie su propia verdad. El modelo del inquirir es la esencia del modelo de justicia impuesto por las instituciones modernas \cite{ernesto-foucault}. 

El Estado moderno no solamente busca una nueva forma de justicia, que es la inquisición o la investigación, sino que también busca una nueva forma de conocimiento. Frente al conocimiento sistemático o escolástico propio de las universidades, la época moderna como una época articulada bajo la razón de Estado busca un \textbf{conocimiento clasificador}. Lo que el Estado quiere saber no es el sistema de todo el mundo, sino que quiere tener clasificados y burocráticamente determinados a los individuos en sus clases (clases sociales, clases de género, pero también clases botánicas, clases zoológicas, etc). Así, según Foucault, el paradigma del conocimiento estatal es la \textbf{enciclopedia}. En la enciclopedia el conocimiento no está estructurado sistemáticamente sino que está perfectamente ordenado alfabéticamente. El empirismo sería una doctrina típicamente estatal que busca conocer la realidad tal y como es y no como debería ser según nuestras propias teorías o según la razón en sentido abstracto \cite{ernesto-foucault}.

Una vez estudiada como funciona la ciencia del Estado, la estadística (también surgen la sociología, la psiquiatría o la economía), Foucault analiza cuáles son los mecanismos que tiene esa sociedad de regularse punitivamente. Según Foucault hay cuatro procedimientos de actividad punitiva, la exclusión, la confiscación, la señalización y el encierro. En la época clásica no se castiga actos sino a personas. Los únicos que están excluidos de este procedimiento son los locos, puesto que los locos no tienen \textit{yo}. La gran pregunta es como es posible que se produjera una evolución de las conductas penales para que de cuatro conductas posibles actualmente se entienda que la forma máxima de la pena es el encierro. La respuesta que nos da Foucault es la siguiente, en la época moderna el delito no se contempla desde un punto de vista moral, es simplemente una violación de la ley civil. La función de la pena ya no es la reinserción o la reinstitución de un orden sagrado, sino simplemente una cuestión preventiva. Sin embargo esto nos plantea una duda, ¿por qué la prisión? Si la función es prevenir, por qué ya no hay ejecuciones expectaculares en público como se hacía antaño. Para responder a ésto Foucault hace un gran recorrido histórico para llegar a la conclusión de que el conjunto de la sociedad es una prisión. Es decir, la prisión no es sino una institución más junto a la Iglesia, la Escuela, la Familia, de normalización y de serialización de la conducta individual \cite{ernesto-foucault}.
% Deleuze
\subsection{Deleuze y las sociedades de control}
% Privacidad pasada
\subsection{Privacidad pasada}
¿Qué fue la privacidad? En Estados Unidos, la privacidad se consideró como el derecho del individuo a determinar con qué extensión debían sus sentimientos, pensamientos y emociones ser comunicados a otros. Nunca debía ser obligado a expresarlos (excepto al ser testigo en un juicio) y en caso de que escogiera expresarlos, debe tener el poder de fijar los límites de la publicidad que se les da. También se expresó como "el derecho a estar solo".

Recalca el autor que esta privacidad pasada que desde nuestra situación actual parece tan idílica estaba lejos de serlo, ya que en Estados Unidos la privacidad ya estaba muerta desde el comienzo para las mujeres y las personas no blancas. El autor cita a la feminista Catharine MacKinnon que modifica la frase expresada en el párrafo anterior a "el derecho del hombre a estar solo", puesto que a veces se ejercía el derecho de la privacidad del hombre a costa de las mujeres. Tampoco existe la privacidad para los menores hasta que cumplen la mayoría de edad, y ciertos grupos como las personas mayores o las personas con discapacidad tienen una privacidad siempre condicionada. Por último, la privacidad tampoco ha existido nunca para la gente pobre, las que no disponen de un hogar o las que no tienen dinero para pagar a un abogado en caso de invasión de la privacidad.
% Dónde estamos
\section{Dónde estamos}
% Tipos medibles
\subsection{Tipos medibles}
% Dividuos e individuos
\subsection{Dividuos e individuos}
% Biopolítica de los dividuos
\subsection{Biopolítica de los dividuos}
% Privacidad presente
\subsection{Privacidad presente}
¿Qué es la privacidad? Para Cheney-Lippold, hoy en día la privacidad no significa desconectarse del mundo o esconder nuestras debilidades para que no las vea un ojo vigilante. Es nuestro derecho a "mantener la distancia sobre otros mediante la preservación de nuestra privacidad corporal y la integridad de nuestro ser", tal como lo expresa el teórico social Anthony Giddens, citado por el autor. 

Dentro del territorio de nuestro ser, sabemos qué se nos ordena, que consecuencias tienen nuestros actos y por tanto podemos determinar lo que significa \textit{ser} nosotros. Sin embargo, si desconocemos las cuestiones listadas anteriormente para con nuestros dividuos, carecemos del derecho de la privacidad tal como describimos al inicio de la sección.

Para los líderes tecnológicos, la privacidad importa bien poco, y no tienen ningún reparo en reconocerlo. Scott McNealy, CEO de Sun Microsystems, expresaba en 1999 que "Tienes cero privacidad de todas formas. Superalo." Para Mark Zuckerberg, creador de Facebook, la privacidad "ya no es una norma social", y para el antiguo CEO de Google Eric Schmidt la privacidad no solo es innecesaria sino que encima es peligrosa, ya que "si estás haciendo algo que no quieres que nadie más sepa, no deberías hacerlo en primer lugar." Por supuesto esta "guerra" contra la privacidad no es más que un enfrentamiento de intereses entre sus negocios y los derechos de la gente, ya que la privacidad les afecta directamente.

Tal como habíamos comentado en la sección \textit{Privacidad pasada}, la privacidad nunca ha existido para ciertos grupos de personas. Sin embargo, esta es la primera vez que los hombres blancos de clase trabajadora se encuentran dentro del control inherente a la vigilancia causada por la invasión de Internet en la vida privada. Ahora bien, según comenta el autor, estas cuestiones de vigilancia y privacidad afectan a nuestro mundo social sin importar el género, raza o clase. El poder nos afecta a todos y es la privacidad la forma que tenemos de regular el cómo lo hace, por lo que es un derecho que cuenta con un gran rango de partidarios.

Esa idea idílica de privacidad realmente no ha muerto del todo, podemos votar (o no), leer un libro en la biblioteca o hablar con quien nos de la gana. Sin embargo, cuando conocemos el contexto en el que estamos, la privacidad nos brinda mucho más que decisiones de acción o inacción, lo que le recuerda al autor al concepto de "privacidad como integridad contextual" del filósofo Helen Nissembaum. Con la privacidad, podemos conocer y entendernos a nosotros mismos, tener una "integridad del ser" como expresaba Anthony Giddens. Mientras que la privacidad liberal abogaba por una "esfera privada", la privacidad como integridad contextual aborda las distintas complejidades de la vida social en red en la cual el poder no nos abandona en ningún momento.

También es la privacidad la que nos permite ser hedonistas, vagos o ignorar las normas sociales en los contextos apropiados, el saber que podemos hacer las cosas de forma distinta en un espacio privado ya que no habrá nadie para juzgarnos. Es gracias a estas desviaciones de la norma por las cuales podemos mantener nuestra integridad del ser.

Para el autor, la privacidad liberal tal y como la conocíamos ya no puede garantizarnos esa protección, es necesaria una nueva forma de privacidad, la privacidad de los dividuos. Cuando estamos compuestos por datos, no tenemos una integridad del ser. Mientras que si andamos por la calle podemos disfrutar de lo que Nissenbaum llama "privacidad en público", es decir, sabemos cómo nuestro cuerpo, nuestro género y nuestra raza generan un sentido que necesitamos para poder saber como vivir, nuestros dividuos no generan un sentido que podamos comprender y por tanto perdemos nuestra integridad del ser.
% Hacia dónde deberíamos ir
\section{Hacia dónde deberíamos ir}
% Privacidad futura
\subsection{Privacidad futura}
¿Qué será la privacidad? La privacidad futura es la privacidad de los dividuos, el erudito en derecho de comunicación Jisuk Woo la define como "el derecho a no ser identificado". Cheney-Lippold recalca que en este caso no ser identificado significa no ser identificado personalmene ni algorítmicamente. Según el erudito en medios de comunicación y el filósofo Hellen Nissembaum este modo de privacidad de la no identificación se práctica de la mejor forma mediante la ofuscación. La ofuscación es la "adición deliberada de información ambigua, confusa o engañosa para interferir con la vigilancia y recopilación de datos".
% Ghost in the shell
\section{Ghost in the shell}
% Bibliografía
\begin{thebibliography}{8}
    \bibitem{ernesto-foucault}
        Clase sobre Michel Foucault impartida por Ernesto Castro, profesor de filosofía de la Universidad Complutense de Madrid, \url{https://youtu.be/uVIt4MX7kvU}. Última vez accedido 18 de enero de 2021.

    \bibitem{ernesto-deleuze}
        Clase sobre Gilles Deleuze impartida por Ernesto Castro, profesor de filosofía de la Universidad Complutense de Madrid, \url{https://youtu.be/Z-iYw9gEpEM}. Última vez accedido 18 de enero de 2021.

    \bibitem{cp-societies}
        Vídeo sobre las sociedades de control, realizado por el graduado en filosofía Jonas Ceika (seudónimo), \url{https://youtu.be/B_i8_WuyqAY}. Última vez accedido 18 de enero de 2021.

    \bibitem{foucault-vs-deleuze}
        Vídeo sobre las sociedades de control, realizado por el estudiante de doctorado de filosofía David Guignion, \url{https://youtu.be/rrqx-zSOTdw}. Última vez accedido 18 de enero de 2021.

    \bibitem{biopolitica}
        Vídeo sobre la biopolítica, realizado por el estudiante de doctorado de filosofía David Guignion, \url{https://youtu.be/MrsJNmwoX6g}. Última vez accedido 18 de enero de 2021.
\end{thebibliography}
\end{document}


